\documentclass[11pt]{article}
\usepackage[a4paper,margin=25mm]{geometry}
\usepackage{amsmath,amssymb,amsthm}
\usepackage{hyperref}
\usepackage{enumitem}

\newtheorem{definition}{Definition}
\newtheorem{theorem}{Theorem}
\newtheorem{proposition}{Proposition}
\newtheorem{remark}{Remark}

\title{Typed Semantics for Root-Spec Integration in UAD/f:\\Lean4 Formalization of U0 Construction}
\author{spec-oracle project}
\date{February 14, 2026}

\begin{document}
\maketitle

\begin{abstract}
We formalize the root-specification integration core (U0 construction) of UAD/f in Lean4.
The research problem is definitional ambiguity: many descriptions write
$U0=\bigcup_i f^{-1}_{0i}(A_i)$ without typing the layer spaces or defining
what $f^{-1}_{0i}$ means for partial mappings.
We introduce a typed setting with root space $\Omega$, per-layer carriers $\beta_i$,
domain predicates $D_i\subseteq \beta_i$, admissible predicates
$A_i\subseteq D_i$, and partial projections
$\mathrm{proj}_i:\Omega\to\mathrm{Option}\,\beta_i$.
Inverse images are induced from \texttt{proj}, not postulated.
We prove: preimage monotonicity and union preservation, domain-respecting lifting
($A_i\subseteq D_i \Rightarrow lifted(i)\subseteq f^{-1}_{0i}(D_i)$),
least-upper-bound characterization of $U0$, monotonic growth under layer-set extension,
a layer-transfer theorem under relation-preserving assumptions,
an inter-layer composition law for induced inverse images,
an extraction-adequacy transfer theorem, and contradiction/consistency duality.
We further report an external case study on real OSS artifacts
(PostgreSQL, zlib, SQLite) using an automated constraint-extraction pipeline.
The contribution is a mechanically checked core semantics for U0 construction,
not full semantic adequacy of the whole UAD/f methodology.
The external study is limited to numeric-bound constraints (not full structural or temporal semantics).
\end{abstract}

\section{Problem Statement and Research Questions}
Layered assurance artifacts are heterogeneous. If a root specification is reconstructed
from layer views, the reconstruction operator must be typed and explicit.
This paper asks:
\begin{enumerate}[leftmargin=2em]
\item[\textbf{RQ1}] Can U0 construction be defined coherently when each layer has its own type?
\item[\textbf{RQ2}] Can admissible-vs-domain constraints be lifted to the root level?
\item[\textbf{RQ3}] Is root construction monotone under layer-set extension?
\item[\textbf{RQ4}] Can a general extraction relation be connected to induced inverse images, and can that theorem be instantiated to a concrete decision rule?
\item[\textbf{RQ5}] Can constraints extracted from real OSS requirement/API/code artifacts be handled consistently in this model?
\end{enumerate}
\noindent
UAD/f is treated as a self-contained model in this paper: all required entities
($\Omega, I, \beta_i, D_i, A_i, \mathrm{proj}_i$) are fully specified in text and Lean code.
\noindent
Paper type: this is a mechanization-oriented original article.
Claims are centered on formal coherence of the typed UAD/f core and reusable machine-checked lemmas; external evidence is limited to a small three-project OSS case study (not broad industrial effectiveness).

\section{Typed UAD/f Core}
\begin{definition}[Typed model]
Fix a root space $\Omega$, index set $I$, and layer carriers $\beta_i$.
For each layer $i$:
\[
D_i \subseteq \beta_i,\qquad A_i \subseteq D_i,\qquad
\mathrm{proj}_i : \Omega \to \mathrm{Option}\,\beta_i.
\]
\end{definition}

\begin{definition}[Induced inverse image]
For $S \subseteq \beta_i$:
\[
f^{-1}_{0i}(S)
:=\{x\in\Omega \mid \exists y\in\beta_i,\ \mathrm{proj}_i(x)=\mathrm{some}(y)\land y\in S\}.
\]
\noindent
In Lean, this is encoded as a predicate \texttt{SpecSet $\Omega$ := $\Omega \to Prop$};
thus the binder ``$x \in \Omega$'' is represented by \texttt{fun x : $\Omega$ => ...}.
\end{definition}

\begin{definition}[Lifted layer and U0]
\[
lifted(i):=f^{-1}_{0i}(A_i),\qquad
U0:=\bigcup_{i\in I} lifted(i).
\]
\end{definition}

\begin{definition}[Inter-layer predicates]
\[
\mathrm{Contradictory}(i,j):\equiv
\forall x,\ x\in lifted(i)\to x\in lifted(j)\to\bot,
\]
\[
\mathrm{Consistent}(i,j):\equiv
\exists x,\ x\in lifted(i)\land x\in lifted(j).
\]
\end{definition}

\section{Main Results}
\begin{theorem}[Preimage monotonicity]
$S\subseteq T \Rightarrow f^{-1}_{0i}(S)\subseteq f^{-1}_{0i}(T)$.
\end{theorem}

\begin{theorem}[Preimage union preservation]
$f^{-1}_{0i}(S\cup T)=f^{-1}_{0i}(S)\cup f^{-1}_{0i}(T)$.
\end{theorem}

\begin{theorem}[Domain-respecting lifting]
$A_i\subseteq D_i \Rightarrow lifted(i)\subseteq f^{-1}_{0i}(D_i)$.
\end{theorem}

\begin{proposition}[Root witness validity]
If $x\in U0$, then $\exists i,\exists y,\ \mathrm{proj}_i(x)=\mathrm{some}(y)\land y\in D_i$.
\end{proposition}

\begin{theorem}[Least-upper-bound characterization]
For any $B\subseteq\Omega$:
\[
(\forall i,\ lifted(i)\subseteq B)\iff U0\subseteq B.
\]
Hence $U0=\bigsqcup_{i\in I} lifted(i)$ in the powerset order.
\end{theorem}

\begin{theorem}[Layer-extension monotonicity]
For active-layer predicates $J,K:I\to\mathrm{Prop}$, if
$\forall i,\ J(i)\to K(i)$ then $U0_J\subseteq U0_K$.
\end{theorem}

\begin{theorem}[Layer-transfer under admissibility-preserving relation]
Let $R : \beta_i \to \beta_j \to \mathrm{Prop}$ satisfy:
(1) each projected $y_j$ has a related projected $y_i$ witness,
(2) $R$ preserves admissibility from $A_j$ to $A_i$.
Then $lifted(j)\subseteq lifted(i)$.
\end{theorem}
\noindent
Mechanized instances are provided in \texttt{TransferExample.lean}
(\texttt{false -> true}) and \texttt{TransferChainExample.lean}
(\texttt{code -> api -> req}).
Operationally, this supports safe propagation of stronger constraints across
traceability updates (e.g., code/API constraints to requirement-layer root witnesses).

\begin{theorem}[Consistency/contradiction duality]
$\mathrm{Contradictory}(i,j)\iff\neg\mathrm{Consistent}(i,j)$.
\end{theorem}

\begin{theorem}[Inter-layer composition law]
If $\mathrm{proj}_j(x)=\mathrm{bind}(\mathrm{proj}_i(x),g)$ for all $x\in\Omega$,
then for every $S\subseteq \beta_j$:
\[
f^{-1}_{0j}(S) = f^{-1}_{0i}(g^{-1}(S)).
\]
\end{theorem}
\noindent
The commutation hypothesis matches extractor pipelines where layer-$j$
projection is implemented as a post-transform over layer-$i$ projection;
see \texttt{CompositionExample.lean}.
This enables decomposition of higher-layer inverse-image checks into lower-layer
checks plus an explicit transformer $g$, useful for incremental extractor validation.

\begin{theorem}[General extraction adequacy transfer]
If an extraction relation $E : \Omega \to \beta_i \to Prop$ satisfies
\[
\mathrm{proj}_i(x)=\mathrm{some}(y) \iff E(x,y),
\]
then for every $S\subseteq \beta_i$:
\[
f^{-1}_{0i}(S)=\{x\in\Omega \mid \exists y,\ E(x,y)\land y\in S\}.
\]
\end{theorem}

\begin{theorem}[Extractor decision equivalence (password policy case)]
For artifacts:
requirement $(min_{req}, max_{req})$, API $(min_{api})$, code $(max_{code})$,
define:
\[
check := \mathbf{decide}\left(
\max(min_{req},min_{api}) \le \min(max_{req},max_{code})
\right).
\]
Then:
\[
check = true \iff \exists n,\ n\in lifted(req)\cap lifted(api)\cap lifted(code).
\]
\end{theorem}

\begin{theorem}[Non-adjunction under partial projection]
If there exists $x_0\in\Omega$ with $\mathrm{proj}_i(x_0)=none$, then
the existential preimage operator at layer $i$ has no left adjoint on full powersets:
\[
\neg \exists F,\ \forall S,T,\ F(S)\subseteq T \iff S\subseteq f^{-1}_{0i}(T).
\]
\end{theorem}

\section{Mechanization and Reproducibility}
Lean artifacts are organized by proof purpose:
\begin{itemize}
\item \texttt{Definitions}: typed UAD/f core (\texttt{Layer}, \texttt{Model}, \texttt{preimage})
\item \texttt{U0Spec}: construction lemmas, LUB theorem, layer-extension monotonicity
\item \texttt{InterLayer}: contradiction/consistency + transfer + composition + adequacy theorem
\item \texttt{CaseStudy}: extractor decision soundness/completeness theorem
\item \texttt{Examples}: satisfiable and contradictory instances
\end{itemize}

Reproduction:
\begin{verbatim}
cd paper/lean
~/.elan/bin/lake build
\end{verbatim}

Pinned toolchain:
\begin{itemize}
\item Lean4: \texttt{leanprover/lean4:v4.27.0}
\item Lake: \texttt{5.0.0-src+db93fe1}
\item toolchain file: \texttt{paper/lean/lean-toolchain}
\item lake definition: \texttt{paper/lean/lakefile.lean}
\item manifest: \texttt{paper/lean/lake-manifest.json} (packages = [])
\item reproducibility commit: \texttt{3b06520d9704c09dcec03d609731910bcfa0edc9}
\end{itemize}

Current footprint:
\begin{itemize}
\item 979 LOC across core Lean files
\item 35 theorem declarations
\end{itemize}

\paragraph{Non-trivial proof engineering points}
\begin{itemize}
\item The dependent carrier family \texttt{carrier : I -> Type} requires explicit witness transport in existential proofs for \texttt{preimage}.
\item The composition theorem relies on case-splitting over \texttt{Option.bind} (\texttt{some}/\texttt{none}) and normalization under the commutation hypothesis.
\item Predicate-as-set encoding (\texttt{SpecSet}) requires explicit membership/subset instances and repeated extensionality reasoning (\texttt{set\_ext}).
\item Quantitatively, \texttt{preimage\_compose} contains two explicit case splits and four \texttt{calc} chains, while \texttt{lifted\_transfer} uses nested witness reconstruction across relation hypotheses.
\item Theorem-level linkage: \texttt{preimage\_union} depends on dependent witness scoping, \texttt{lifted\_subset\_preimage\_domain} on explicit projection witnesses, and \texttt{preimage\_compose} on \texttt{Option.bind} normalization.
\end{itemize}

\paragraph{Synthetic consistency cross-check}
We provide a reproducible benchmark (\texttt{paper/case-study/password\_policy\_benchmark.py})
as a consistency check (not a proof substitute) with 5 seeds $\times$ 200,000 generated artifact triples.
The key validation is agreement between the closed-form decision rule and brute-force witness search:
\begin{itemize}
\item baseline comparison (20,000 cases): mismatches = 0
\item witness validity check (20,000 cases): witness\_violation\_count = 0
\item distribution indicator: contradiction ratio = 0.214478
\end{itemize}
\noindent
This synthetic check validates implementation alignment only; external validity is evaluated separately in the next paragraph.

\paragraph{External validation on real OSS artifacts}
We add a real-artifact case study
(\texttt{paper/case-study/real\_projects/external\_validation.py})
that fetches official docs/source pages and extracts layer constraints:
\begin{itemize}
\item PostgreSQL identifier length (runtime preset / lexical doc / \texttt{NAMEDATALEN})
\item zlib compression level (manual / Python API doc / \texttt{zlib.h} constants)
\item SQLite page size (PRAGMA doc / file-format doc / \texttt{sqliteLimit.h})
\end{itemize}
Observed results (2026-02-14):
\begin{itemize}
\item real projects: 3, consistent: 3, contradictory: 0
\item mutation sensitivity test (forcing requirement lower bound above upper bound): 3/3 contradictions detected
\end{itemize}
This section is the main case-study evidence for external applicability at small scale.
For RQ5, these values are the primary external-validity indicators:
\texttt{n\_real\_projects=3}, \texttt{n\_real\_consistent=3}, mutation detection \texttt{3/3}.
Their theoretical role is to instantiate Theorem 3 (domain-respecting lifting)
and Theorem 4 (existence of domain-valid witnesses) on non-synthetic artifacts.

\section{Related Work and Positioning}
Our order-theoretic lemmas are compatible with classical abstract interpretation
viewpoints (Cousot \& Cousot, 1977), while our target is narrower:
typed partial projections for root-specification reconstruction.
In particular, our \texttt{preimage} operator is induced by
$\mathrm{proj}:\Omega\to Option\,\beta_i$; because undefined points are explicit,
embedding into a standard total-map Galois adjunction requires additional domain/totality assumptions.
This difference is formalized by a mechanized non-adjunction theorem for undefined projection points.
This is not an isolated fact: it justifies why U0 correctness arguments in this
paper explicitly track witnesses (\texttt{proj\_i(x)=some(y)}) instead of relying on
implicit adjunction-based transport.
Institution-style semantic transport (Goguen \& Burstall, 1984) addresses
language-independent satisfaction, whereas we focus on a concrete U0 operator.
Our correspondence is: layer index set $I$ as signature-level indexing and
$lifted(i)$ as root-level satisfaction set for layer $i$.
Unlike general institution axiomatization, satisfaction-preservation obligations
are given explicitly as assumptions ($R$ in transfer, $hcomm$ in composition).
Compared with system-level formal methods (e.g., TLA+, Lamport 2002), this work
does not model full system behavior; it isolates and mechanizes the U0 integration kernel,
and connects it to an executable decision rule in a concrete case study.

\section{Scope and Limitations}
\begin{remark}
This paper proves internal coherence of typed U0 construction.
It does \emph{not} prove full adequacy of UAD/f in practical engineering pipelines.
\end{remark}

Not proven here:
\begin{enumerate}[leftmargin=2em]
\item faithfulness of concrete extractors to $\mathrm{proj}_i$,
\item global completeness/soundness of end-to-end UAD/f workflows,
\item large-scale external validity (current real-artifact study covers three projects and numeric-bound constraints).
\end{enumerate}

\section{Conclusion}
We remove ambiguity in U0 construction by giving a typed UAD/f core and
defining $f^{-1}_{0i}$ via partial projections.
Lean4 proofs establish domain-respecting lifting, supremum characterization of U0,
layer-extension monotonicity, composition preservation, and extraction-adequacy transfer.
This provides a precise baseline for stronger future results (composition laws,
extractor correctness, and CI-level artifact guarantees).
The scientific increment is not restating elementary set facts, but exposing
missing assumptions in typed partial multi-layer integration (including mechanized
non-adjunction under undefined projections).

\paragraph{References (informal)}
[1] P. Cousot and R. Cousot, ``Abstract Interpretation,'' POPL 1977.\\
[2] J. A. Goguen and R. M. Burstall, ``Introducing Institutions,'' 1984.\\
[3] L. Lamport, \emph{Specifying Systems}, 2002.

\end{document}
