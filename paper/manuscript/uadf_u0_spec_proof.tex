\documentclass[11pt]{article}
\usepackage[a4paper,margin=25mm]{geometry}
\usepackage{amsmath,amssymb,amsthm}
\usepackage{hyperref}
\usepackage{enumitem}

\newtheorem{definition}{Definition}
\newtheorem{theorem}{Theorem}
\newtheorem{proposition}{Proposition}
\newtheorem{remark}{Remark}

\title{Typed U0 Construction in the UAD/f Model:\\A Lean4 Mechanization}
\author{spec-oracle project}
\date{February 14, 2026}

\begin{document}
\maketitle

\begin{abstract}
We formalize the U0-construction core of the UAD/f model in Lean4.
The key issue is definitional ambiguity: many descriptions write
$U0=\bigcup_i f^{-1}_{0i}(A_i)$ without typing the layer spaces or defining
what $f^{-1}_{0i}$ means for partial mappings.
We introduce a typed setting with root space $\Omega$, per-layer carriers $\beta_i$,
domain predicates $D_i\subseteq \beta_i$, admissible predicates
$A_i\subseteq D_i$, and partial projections
$\mathrm{proj}_i:\Omega\to\mathrm{Option}\,\beta_i$.
Inverse images are induced from \texttt{proj}, not postulated.
We prove: preimage monotonicity and union preservation, domain-respecting lifting
($A_i\subseteq D_i \Rightarrow lifted(i)\subseteq f^{-1}_{0i}(D_i)$),
least-upper-bound characterization of $U0$, monotonic growth under layer-set extension,
a layer-transfer theorem under relation-preserving assumptions,
and contradiction/consistency duality.
The contribution is a mechanically checked core semantics for U0 construction,
not full semantic adequacy of the whole UAD/f methodology.
\end{abstract}

\section{Problem Statement and Research Questions}
Layered assurance artifacts are heterogeneous. If a root specification is reconstructed
from layer views, the reconstruction operator must be typed and explicit.
This paper asks:
\begin{enumerate}[leftmargin=2em]
\item[\textbf{RQ1}] Can U0 construction be defined coherently when each layer has its own type?
\item[\textbf{RQ2}] Can admissible-vs-domain constraints be lifted to the root level?
\item[\textbf{RQ3}] Is root construction monotone under layer-set extension?
\item[\textbf{RQ4}] Can a concrete extractor-side decision rule be proved equivalent to the logical consistency predicate?
\end{enumerate}

\section{Typed UAD/f Core}
\begin{definition}[Typed model]
Fix a root space $\Omega$, index set $I$, and layer carriers $\beta_i$.
For each layer $i$:
\[
D_i \subseteq \beta_i,\qquad A_i \subseteq D_i,\qquad
\mathrm{proj}_i : \Omega \to \mathrm{Option}\,\beta_i.
\]
\end{definition}

\begin{definition}[Induced inverse image]
For $S \subseteq \beta_i$:
\[
f^{-1}_{0i}(S)
:=\{x\in\Omega \mid \exists y\in\beta_i,\ \mathrm{proj}_i(x)=\mathrm{some}(y)\land y\in S\}.
\]
\end{definition}

\begin{definition}[Lifted layer and U0]
\[
lifted(i):=f^{-1}_{0i}(A_i),\qquad
U0:=\bigcup_{i\in I} lifted(i).
\]
\end{definition}

\begin{definition}[Inter-layer predicates]
\[
\mathrm{Contradictory}(i,j):\equiv
\forall x,\ x\in lifted(i)\to x\in lifted(j)\to\bot,
\]
\[
\mathrm{Consistent}(i,j):\equiv
\exists x,\ x\in lifted(i)\land x\in lifted(j).
\]
\end{definition}

\section{Main Results}
\begin{theorem}[Preimage monotonicity]
$S\subseteq T \Rightarrow f^{-1}_{0i}(S)\subseteq f^{-1}_{0i}(T)$.
\end{theorem}

\begin{theorem}[Preimage union preservation]
$f^{-1}_{0i}(S\cup T)=f^{-1}_{0i}(S)\cup f^{-1}_{0i}(T)$.
\end{theorem}

\begin{theorem}[Domain-respecting lifting]
$A_i\subseteq D_i \Rightarrow lifted(i)\subseteq f^{-1}_{0i}(D_i)$.
\end{theorem}

\begin{proposition}[Root witness validity]
If $x\in U0$, then $\exists i,\exists y,\ \mathrm{proj}_i(x)=\mathrm{some}(y)\land y\in D_i$.
\end{proposition}

\begin{theorem}[Least-upper-bound characterization]
For any $B\subseteq\Omega$:
\[
(\forall i,\ lifted(i)\subseteq B)\iff U0\subseteq B.
\]
Hence $U0=\bigsqcup_{i\in I} lifted(i)$ in the powerset order.
\end{theorem}

\begin{theorem}[Layer-extension monotonicity]
For active-layer predicates $J,K:I\to\mathrm{Prop}$, if
$\forall i,\ J(i)\to K(i)$ then $U0_J\subseteq U0_K$.
\end{theorem}

\begin{theorem}[Layer-transfer under admissibility-preserving relation]
Let $R : \beta_i \to \beta_j \to \mathrm{Prop}$ satisfy:
(1) each projected $y_j$ has a related projected $y_i$ witness,
(2) $R$ preserves admissibility from $A_j$ to $A_i$.
Then $lifted(j)\subseteq lifted(i)$.
\end{theorem}

\begin{theorem}[Consistency/contradiction duality]
$\mathrm{Contradictory}(i,j)\iff\neg\mathrm{Consistent}(i,j)$.
\end{theorem}

\begin{theorem}[Extractor decision equivalence (password policy case)]
For artifacts:
requirement $(min_{req}, max_{req})$, API $(min_{api})$, code $(max_{code})$,
define:
\[
check := \mathbf{decide}\left(
\max(min_{req},min_{api}) \le \min(max_{req},max_{code})
\right).
\]
Then:
\[
check = true \iff \exists n,\ n\in lifted(req)\cap lifted(api)\cap lifted(code).
\]
\end{theorem}

\section{Mechanization and Validation}
Lean artifacts are organized by proof purpose:
\begin{itemize}
\item \texttt{Definitions}: typed UAD/f core (\texttt{Layer}, \texttt{Model}, \texttt{preimage})
\item \texttt{U0Spec}: construction lemmas, LUB theorem, layer-extension monotonicity
\item \texttt{InterLayer}: contradiction/consistency + transfer theorem
\item \texttt{CaseStudy}: extractor decision soundness/completeness theorem
\item \texttt{Examples}: satisfiable and contradictory instances
\end{itemize}

Reproduction:
\begin{verbatim}
cd paper/lean
~/.elan/bin/lake build
\end{verbatim}

Pinned toolchain:
\begin{itemize}
\item Lean4: \texttt{leanprover/lean4:v4.27.0}
\item Lake: \texttt{5.0.0-src+db93fe1}
\end{itemize}

Current footprint:
\begin{itemize}
\item 649 LOC across core Lean files
\item 27 theorem declarations
\end{itemize}

\paragraph{Synthetic empirical run}
We provide a reproducible benchmark (\texttt{paper/case-study/password\_policy\_benchmark.py})
with 5 seeds $\times$ 200,000 generated artifact triples.
Observed averages:
\begin{itemize}
\item contradiction ratio: 0.214478
\item elapsed time: 0.7499 sec / 200,000 cases
\item throughput: 267,334 cases/sec
\end{itemize}

\section{Related Work and Positioning}
Our order-theoretic lemmas are compatible with classical abstract interpretation
viewpoints (Cousot \& Cousot, 1977), while our target is narrower:
typed partial projections for root-specification reconstruction.
Institution-style semantic transport (Goguen \& Burstall, 1984) addresses
language-independent satisfaction, whereas we focus on a concrete U0 operator.
Compared with system-level formal methods (e.g., TLA+, Lamport 2002), this work
does not model full system behavior; it isolates and mechanizes the U0 integration kernel,
and connects it to an executable decision rule in a concrete case study.

\section{Scope and Limitations}
\begin{remark}
This paper proves internal coherence of typed U0 construction.
It does \emph{not} prove full adequacy of UAD/f in practical engineering pipelines.
\end{remark}

Not proven here:
\begin{enumerate}[leftmargin=2em]
\item faithfulness of concrete extractors to $\mathrm{proj}_i$,
\item global completeness/soundness of end-to-end UAD/f workflows,
\item scalability/complexity bounds for large repositories.
\end{enumerate}

\section{Conclusion}
We remove ambiguity in U0 construction by giving a typed UAD/f core and
defining $f^{-1}_{0i}$ via partial projections.
Lean4 proofs establish domain-respecting lifting, supremum characterization of U0,
layer-extension monotonicity, and contradiction/consistency duality.
This provides a precise baseline for stronger future results (composition laws,
extractor correctness, and CI-level artifact guarantees).

\paragraph{References (informal)}
[1] P. Cousot and R. Cousot, ``Abstract Interpretation,'' POPL 1977.\\
[2] J. A. Goguen and R. M. Burstall, ``Introducing Institutions,'' 1984.\\
[3] L. Lamport, \emph{Specifying Systems}, 2002.

\end{document}
